\documentclass[a4paper,12pt]{article}

\title{CANTHERM}
\author{Sandeep Sharma}
\begin{document}
 \maketitle
\tableofcontents
\section{Introduction}
CANTHERM is written in python and calculates the themodynamic properties of stable molecules and rate coefficients for reactions. The types of calculations it can perform are
\begin{itemize}
 \item Calculates enthalpy, entropy and heat capacities using Rigid Rotor Harmonic Oscillator approximation with correction for hindered rotors.
\item Calculates the high pressure rate coefficients for a unimolecular or a bimolecular reaction 
\end{itemize}
The manual first attempts to give some background theory and then goes on to give the instructions for making the input file. Finally a couple of example input files along with the output are given.

\section{Theory}
Rigid Rotor Harmonic oscillator assumes separability of different modes of vibrations and interal/external rotations in a molecule. Each mode is then treated separately to calculate the parition function and subsequently the thermodynamic quantities from the partition functions. We treat each of the modes separately.
\subsection{Translation}
The parition function for translation is given in Equation~\ref{eq:tr1}. The resulting thermodynamic functions are given in Equations~\ref{eq:tr2}
\begin{equation}
 Q_{tr} = \left(\frac{2\pi m k_bT}{h^2}\right)^{3/2}V
\label{eq:tr1}
\end{equation}

\begin{eqnarray}
 S%&=&\frac{\partial}{\partial T}\left(k_bT\ln Q_{tr}\right)\nonumber\\
&=&k_b \ln\left[\left(\frac{2\pi m k_bT}{h^2}\right)^{3/2} \frac{k_bTe^{5/2}}{p}\right]\nonumber\\
c_p&=& \frac{5}{2}k_b \nonumber \\
\Delta H&=& \frac{5}{2}k_bT
\label{eq:tr2}
\end{eqnarray}

\subsection{Vibrations}
Vibrational parition functions and the corresponding thermodynamic functions can be calculated exactly assuming that the vibrations behave like a harmonic oscillator. If the frequency of the $i^{th}$ vibration is given by $/nu_i$ then the quantities of interest are given in Equation~\ref{eq:vi1}. Note that for the expression of $\Delta H$ we have not included the zero point energy because we assume that the ground state is the electronic energy plus the zero point energy. It is the same reason that the $\exp{(-h\nu_i/2k_bT)}$ term is missing from the numerator of the parition function. The reference energy has no effect on the heat capacity and entropy and their formulas will remain the same.
\begin{eqnarray}
Q_{ho} &=& \frac{1}{1-\exp{(-h\nu_i/k_bT)}} \nonumber \\
S &=& -k_b \ln\left(1-\exp{(-h\nu_i/k_bT)}\right)+\frac{h\nu_i}{T}\frac{1}{\exp{(h\nu_i/k_bT)}-1}\nonumber \\
cp &=& k_b\left(\frac{h\nu_i}{k_bT}\right)^2 \frac{\exp{(h\nu_i/k_bT)}}{\left(\exp{(h\nu_i/k_bT)}-1\right)^2}\nonumber \\
\Delta H &=& \frac{h\nu_i}{\exp{(h\nu_i/k_bT)}-1}
\label{eq:vi1}
\end{eqnarray}

\subsection{Rotations}
For the molecule without internal rotors only the external rotors contribute to the rotational parition function. The partition functions and thermodynamic quantities for these rotors is relatively straight forward to calculate and are given in Equations~\ref{eq:r1}.
\begin{eqnarray}
 Q_{e,r}&=&\frac{\pi^{1/2}}{\sigma_{ext}}\prod_i \left(\frac{8\pi^2I_ik_bT}{ h^2}\right)^{1/2} \nonumber \\
S&=& k_b \left[\frac{\pi^{1/2}}{\sigma_{ext}} \ln \left(\prod_i \frac{8\pi^2I_ik_bT}{h^2}\right)^{1/2} +\frac{3}{2}\right] \nonumber \\
c_p&=& \frac{3}{2}k_b\nonumber \\
\Delta H &=& \frac{3}{2}k_bT
\label{eq:r1}
\end{eqnarray}

When a molecule has internal rotors then the external and internal rotors are coupled to each other and the combined parition function is calculted using the semi-classical Pitzer-Gwinn approximation given in Equation~\ref{eq:1}, where $Q_{r}$ is the total rotational parition function including the internal and external modes, $Q_{r,cl}$ is the classical parition function for the combined internal and external rotations calculated using the phase space integral, $Q_{ho,q}$ is the quantum partition function of the vibrational modes corresponding to the hindered rotor and $Q_{ho,cl}$ is the classical partition function of the vibrational modes corresponding to the hindered rotor.
\begin{equation}
 Q_{r} = Q_{r,cl}\frac{Q_{ho,q}}{Q_{ho,cl}}
\label{eq:1}
\end{equation}

The expressions for each of the parition functions is given in Equation~\ref{eq:2} and Equation~\ref{eq:vi1}.
\begin{eqnarray}
 Q_{r,cl} &=& \frac{\pi^{1/2}}{\sigma_{ext}}\left(\frac{8\pi^2k_bT}{ h^2}\right)^{3/2} \left( \frac{2 \pi k_b T}{h^2}\right)^{n/2} \frac{1}{\prod_i \sigma_i}\int_{\Phi} [D]^{1/2} \exp{(-V(\Phi)/k_bT)} d\Phi \nonumber \\
Q_{ho,cl} &=& \prod_i \frac{k_bT}{h\nu_i}
\label{eq:2}
\end{eqnarray}

For free rotors we just substitute $V(\Phi)=0$ in the expression for $Q_{r,cl}$ to get the expression shown in Equation~\ref{eq:3}.
\begin{equation}
 Q_{fr,cl} = \frac{\pi^{1/2}}{\sigma_{ext}}\left(\frac{8\pi^2k_bT}{ h^2}\right)^{3/2} \left( \frac{2 \pi k_b T}{h^2}\right)^{n/2} \frac{1}{\prod_i \sigma_i}\int_{\Phi} [D]^{1/2} d\Phi
\label{eq:3}
\end{equation}

Usually when only free rotors are present in the molecule one would just pick the geometry of the most stable confirmer and calculate the kinetic energy matrix $D$. In Cantherm we use the procedure described in 1949 paper of Pitzer and Gwinn which is described as $I^{m=5}$ in the paper by East and Radom. This method is completely accurate for a molecule which has rigid frames moving with respect to each other. The rigid frame approximation ignores the rotational-vibrational coupling in the molecule. 

To calculate the phase space integral in Equation~\ref{eq:2} there are a few approximations that can be made. If we assume that the potential $V(\Phi)$ can be broken up as a sum of potentials in different phase angles i.e. $V(\Phi) = V_1(\phi_1)+V_2(\phi_2)...$ then the multidimensional integral can be broken into a product of single dimensional integrals. These integrals can then separately be evaluated using trapezoidal rule or monte-carlo integration. For each phase a complete scan is performed and then the scan potential is fit to a fourier series of the form $V_i(\phi_i)=\sum_m \cos(m\phi_i)+\sin(m\phi_i)$. The fitted parameters can then be used to calculate the integrals.

For some rare cases the separation of variables does not work very well. In this case the general approach is to calculate the multidimensional integral numerically using monte-carlo integration technique. Each of the phase angle $\phi_i$ is chosen randomly and for that chosen phase angle the energy of the molecule is evaluated using some ab-initio method.
 
\subsection{Identification of internal rotation modes}
For small molecules the identification of internal modes of rotation is faily straight forward. If a viewing package is used one of the normal modes usually very strongly resembles an internal rotation. But for larger molecules the situation becomes complicated the normal modes of vibrations have many internal rotational modes along with other modes including bending and stretching modes mixed in. In this case identifying the correct normal mode to remove and treat it as hindered rotor can be tricky. For such cases evaluating the force constant matrix and projecting out the force constants along the internal rotation coordinate can give us a more accurate picture. 

Let us assume that atom 1 and atom 2 are the two pivot atoms for an internal rotor $i$. Also assume that a list of atoms $k$ is attached to pivot atom 1 and a list of atoms $l$ is attached to pivot atom 2. Then the instantaneous displacement vector for an atom $kj$ in list $k$ is given by the vector in Equation~\ref{eq:dis}, where $p_n$ is the position vector of the $n^{th}$ atom.
\begin{equation}
s_{kj} = \frac{(p_{kj}-p_1) \times (p_2-p_1)}{\left|p_2-p_1\right|}
\label{eq:dis}
\end{equation}

Similar displacement vectors can be calculated for each atom and then the $i^{th}$ internal rotation in cartesian coorinate is given by vector $v_i = [s_1 s_2 s_3 ... s_n]^T$ where $s_i$ is the instantaneous displacement vector give in Equation~\ref{eq:dis} and $n$ is the total number of atoms in the molecule. One the vectors $v_i$ corresponding to all the internal rotors are calculated an orthonormal set $w_i$ is generated from these vectors. The projection vector $P$ can be generated from these orthonormal set $w_i$ as shown in Equation~\ref{eq:pr} where matrix $W=[w_1 w_2..]$ has column vectors $w_i$.

\begin{equation}
 P = WW^T
\label{eq:pr}
\end{equation}

If the complete force constant matrix is $Fc$ then the new force constant matrix $Fcr$ from which the force constants along the internal rotation vectors are removed are given by Equation~\ref{eq:frc} and has $m$ less non-zero (or nearly non-zero) eigenvalues as $Fc$, where $m$ is the number of internal rotors.

\begin{equation}
 Fcr = (I-P)Fc(I-P)
\label{eq:frc}
\end{equation}

This force matrix $Fcr$ can be converted into mass-weighted cartesian coordinates and diagonalized to get the eigenvalues from which the vibrational frequencies can be calculated. 
 
\section{Input File}
\begin{itemize}
 \item The first line of the file contains the keywords describing the job type to be performed, the two options are \textbf{Thermo} or \textbf{Rate}. As the name suggets the thermo keyword is used to calculated the thermodynamic quantities including entropy, heat capacity and thermal correction for a given molecule. The Rate keyword is used to calculate the rate coefficients for a reaction and also the fitted Arrhenius parameters.
	
If the keyword \textbf{Rate} is used then the next line has to describe whether the reaction is unimolecular or bimolecular by giving keywords \textbf{Unimol} or \textbf{Bimol} respectively.

\item The next line gives the range of Temperatures at which the thermochemistry or the rates are to be evaluated. An example input line would be \textbf{Trange 300 100 15}, which tells CANTHERM that the temperatures at which calculations are to be performed are given by $T_i = T_0 + i*dT$, where $T_0=300$, $i$ takes values from 0 to $n=15$ and $dT=15$, all values are in Kelvin.

\item The next line is \textbf{Mol 1} which is followed by information of molecule 1.
\item The next line is \textbf{Atom} or \textbf{Linear}  or \textbf{Nonlinear}, where each of the descriptors stand for the molecule in question.
\item The next line is \textbf{Atoms 14}, here 13 represents the number of atoms in a molecule
\item The next line is \textbf{Geom }, and it is followed in the next few lines by the geometry of the molecule. Each line contains the atomic number of the atom followed by x, y and z coordinates of the atom. 
\item The next line gives information about hindered rotors. The first word is \textbf{Hindrot}, the next word is the number of hindered rotors , e.g. \textbf{5} and the next 5 lines give information about hindered rotors. To specify the hindered rotors one starts by assuming a rigid frame. This rigid frame will have rotors attached to it. Each rotor can in turn have other rotors attached to them. The rigid frame itself is given level 1 and contains all the atoms in the molecule. For example if we take an example of 

\end{itemize}

\end{document}
